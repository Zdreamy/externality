\begin{englishabstract}

	With the popularity of mobile devices and the rapid development of related fields such as mobile social networks and online entertainment, subscribed mobile users and data traffic around the world have experienced a trend of substantial growth in recent years. The enlarging demands for high-quality communication and data service give rises to new challenges to the development of wireless network technology. To optimize the overall network performance, the network operators can rely on the enhancement on network infrastructure and capabilities brought by the next-generation 5G technology. In addition, they ought to exploit the limited resource in an efficient and reasonable manner via sophisticated designed resource allocation mechanism and service pricing mechanism. On the other hand, with the growing sensing capabilities of mobile devices, the mobile network has become the main channel for third-party applications to collect users' personal data and urban environment data, which has promoted the development of mobile big data technologies. While on the other side of the coin, the exposure of personal data has incurred serious privacy crisis and raised people's concern about the risk of privacy leakage.
	
	As one of the main topics in the research field of network economics, network externality is defined as the indirect impact on an individual's utility from other individuals in the rest of the network. Such kind of effect may simply depend on the size of the network, or depend on the degree of involvement of other individuals, as well as the strength of the association between individuals. In recent years, the continuous expansion on the network-scale has made the phenomenon of network externalities increasingly prominent in some scenarios. Specifically, the utility obtained by an individual user from the network can be increased (with positive network externality) or decreased (with negative network externality) by the enhanced involvement of other users in the network. It is due to such prominent influences that network externality has on the individuals' utility, taking the network externality into consideration is of great importance. 
In recent years, researchers have made a lot of progress in the protocol designs aiming for network performance optimization and privacy protection. However, there are still three aspects that require more efforts to be taken: 1) the lack of analysis and investigation on solving network optimization from the view of network externality; 2) the lack of research work combining network performance optimization and users' privacy protection, especially the characterization on the trade-off between the system performance and the privacy; 3) the lack of heterogenous modeling about individuals' strategic behaviors. Based on the state-of-the-art, this thesis has made an effort on network performance optimization within three scenarios: the spectrum sharing network, the mobile crowdsensing network, and the mobile data service market. The main contributions are summarized as follows: 
	\begin{enumerate}
		\item A brief introduction on the background of network economics and the basic concepts of network externality and some discussion on the related works about mechanism design and privacy protection in mobile networks, is provided.
    		\item The first part studies the cost minimization problem of privacy-preserving mobile crowdsensing system and proposes an auction based framework for privacy-preserving data aggregation. Given the heterogeneity on mobile users' sensing capabilities and their unit privacy cost, the platform is facing the challenges of determining the set of mobile users for which the noise distribution need to be carefully designed so that the privacy protection is provided and the data aggregation accuracy is satisfied. This work carries out a local data perturbation scheme and reveal the negative externality concerning the users' privacy-preserving level. And two different settings, ``privacy passive'' setting and ``privacy proactive'' setting, are discussed separately which corresponds to two kinds of privacy-preserving attitudes. Based on the hidden monotonicity of the problem, two computational efficient incentive mechanisms, DPDA and EDPDA, are proposed satisfying the truthfulness and individual rationality properties. By using the the proposed algorithms, the platform can approximately minimize the cost under the aggregation accuracy constraint in both two settings. The performance of the algorithm is validated through extensive numerical experiments.
    		\item The second part studies the socially-aware throughput maximization problem within the context of locational privacy-preserving database-assisted spectrum sharing. The mobile users jointly take into account their physical coupling (negative externality due to the signal interference) and social coupling (positive externality due to the social network effect) while making spectrum sharing decisions. In particular, to mitigate RSS-based PHY-layer location privacy threat, a power perturbation approach is employed where each secondary user judiciously ``reduces'' its transmission power by choosing a power level following a statistical distribution (with a negative bias). Accordingly, the privacy-preserving spectrum sharing among users is cast as a stochastic channel selection game, where strategic players (secondary users) adjust their strategies dynamically aiming to maximize their social group utilities. Specifically, a two-time-scale distributed learning algorithm based on no regret-based rule is devised, which is shown to converge almost surely towards the set of socially-aware correlated equilibrium. The numerical results corroborate that the higher the privacy protection level, the more significant the degradation of the network throughput would be.
    		\item The third part studies the revenue maximization problem for the wireless service providers. The pricing strategies of multiple service providers in a competitive data service market are studied where mobile users' data consumption behaviors are influenced by two effects: the network effect (positive network externality) and the congestion effect (negative network externality). To analyze the strategic interactions between mobile users and service providers, a two-stage Stackelberg game is devised, consisting of a providers' game in Stage I and a users' game in Stage II, respectively. In particular, for the users' game, the equilibrium solution is characterized explicitly and its uniqueness is established. For the providers' game, the analysis indicates that a mixed-strategy equilibrium solution is guaranteed for the scenario with rational providers as well as the scenario with providers of bounded rationality. A distributed learning algorithm for finding a mixed-strategy equilibrium solution is further provided. And the numerical results provide insights into how positive network effect and congestion effect would impact the system performance, and demonstrate that the bounded rational behavior incurs degradation to service providers' revenues.
    		\item The conclusions are drawn with future work at the end of the dissertation.
	\end{enumerate}

	\englishkeywords{Network Externality; Network Performance Optimization; Mechanism Design; Privacy Protection; Game Theory}

\end{englishabstract}
