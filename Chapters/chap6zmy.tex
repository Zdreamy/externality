\chapter{总结与展望}
\textbf{本章摘要:} 本章总结了全文的主要工作和主要结论,对需要进一步研究的方向和问题进行了展望。

\section{全文总结}
%\rv{本文旨在研究毫米波通信系统中资源分配与优化问题。毫米波有着丰富的非授权频谱资源,可以极大提高通信速率,解决现有无线网络系统中频谱资源紧张等问题,为下一代无线通信系统提供了关键的技术支撑,拥有着巨大的应用潜力。毫米波信号在空间中衰减很快,需要通过Massive MIMO天线阵列结合波束成形技术将信号集中成极窄的方向性波束进行传输。新的技术为无线通信系统带来了无限可能,同时也带来了新的挑战,如何在资源受限的情况下提高系统各项性能便成为了毫米波通信系统中急需解决的重要问题。}

%移动通信和移动互联网的发展给人们的生活带来了巨大的改变,人们日常生活中的学习、工作、娱乐、社交活动已经与移动设备紧密地结合在一起。近年来,5G通信技术的成熟大大促使了移动流量的增长,移动社交网络的盛行则使得移动网络呈现出显著的社交效应。然而,由于频谱资源、网络容量的有限性,通信信号干扰、网络拥塞等问题仍然是阻碍移动通信网络发展的主要挑战。从网络经济学的角度,移动网络中的社交效应以及信号干扰、网络拥塞都是网络外部性在移动网络中的具体表现。现有的网络性能优化与机制设计的工作中仍然缺乏对于网络外部性效应的充分挖掘与分析,没有对于个体非完全理性的行为模型进行充分的考量,同时缺乏对于网络性能优化与隐私保护之间权衡的分析。本文针对现有相关研究工作中的不足,结合国内外最新的研究成果,考虑了不同场景中网络外部性影响下移动网络机制设计、性能优化与隐私保护的问题,对不同场景中的网络外部性效应进行了刻画,提出了相应的机制设计以及求解算法。本文的主要结论可以概括为以下几个方面:

信息网络的出现引发和推进了人们生活生产方式和经济社会形态的深刻变革。其中移动网络以其自身的特点和优势,已成为人们学习、工作、生活和经济社会发展离不开的崭新空间。随着经济社会的不断发展和进步,人们对移动网络的服务质量的要求日益增高、对隐私保护和安全风险的担忧日益加深。因此,对于移动网络效用优化机制的创新设计愈发凸显出其重要性与现实意义。
%研究者们针对移动网络中的性能效用优化及机制设计研究取得了很多进展,然而现有研究难免存在诸多不足。比如:在解决网络效用优化问题时对网络外部性的作用认识不够充分,缺乏从网络外部性角度的深入分析和讨论;优化网络性能与用户隐私保护相结合的研究分析还较少,缺乏针对性能提升—隐私保护之间权衡的考量与刻画;现有激励机制设计中对于用户个体在交互中的策略性行为假设还过于理想,与实际情况存在差距导致其效果也有明显局限性。这些问题都有待作进一步的深化细化研究。
针对现有相关研究的不足,本文在国内外相关问题的研究基础上,围绕提升移动网络运行中各主体的整体效用,以网络外部性为重要视角,对移动网络中群智感知、频谱共享、数据服务三个基本场景中的网络效用优化问题进行了基础理论层面的研究。在这三个场景中,本文相应地提出了激励相容机制、频谱共享机制、数据定价机制及隐私保护机制设计的改进方案及协议算法。本文的研究成果丰富了移动网络效用优化理论,对移动网络机制设计实践具有指导意义。本文的创新之处和主要贡献有:

\begin{enumerate}
%    \item 研究了在毫米波无线通信系统中多用户跟踪实时性问题。毫米波信号需要基站通过极窄的方向性波束与用户进行通信。获取用户精确的位置信息能够帮助更好地进行波束对准,因此基站需要实时准确地感知并估计小区内的目标个数及其各自的位置信息,即一个多目标跟踪实时性问题。现有的高效多目标跟踪算法难以做到流水线化计算,制约了实时性的进一步提高。本文提出了一种改进的重采样算法,通过引入粒子复制序列集合及待复制粒子列表,使得需要重采样的粒子在只获得自身及之前重采样粒子信息时即可进行粒子复制运算。通过引入此改进重采样算法,整个多目标概率假设密度粒子滤波器能够实现完全流水线化运算。在此基础上,本文提出了基于多核处理器平台的资源分配优化问题,通过解决一系列混合整数规划问题,得到了高时效的处理器资源分配方案。仿真结果验证了提出的算法在能保证跟踪精度的前提下,显著降低整个滤波算法的计算时延,有效地提高了用户跟踪的实时性,为基站与每个用户的低时延通信提供了保证。此项工作为之后的毫米波通信过程提供了实时准确的用户位置信息,是毫米波通信系统过程中不可或缺的部分,具有重要的研究意义。
%    \item 在实时得到用户准确位置信息的基础上,以蜂窝小区基站为例研究了毫米波通信在星型网络中的资源优化分配问题。提出了基站在已知小区内多用户位置的前提下,如何分配其巨大数量的天线阵列资源以最大化系统收益的方法。基站上的Massive MIMO天线阵列虚拟的分成若干个均匀线性子阵列,分别利用波束成形技术与对应的用户进行通信。由于大规模天线阵列与每个均匀线性子阵列都有着一定的形状,因此问题建模为如何在矩形阵列中得到合适的线性子阵列分配及放置方式,使得系统吞吐量最大。这不仅需要考虑每个用户所对应子阵列的天线数量,还需要考虑这些线性子阵列在矩形阵列中的位置分配。此场景中按照放置方式不同又包含两种情况:1)所有线性子阵列都互相平行于矩形阵列的一条边;2)为了进一步提高系统收益,线性子阵列可以相互垂直放置。针对两个NP-hard问题,将每种情况都进行子问题分解并逐步解决。通过解决一系列的多选择背包问题、多背包问题和带状装箱等组合优化问题,得到了两种情况下的多项式时间近似算法,并给出了每个算法的计算复杂度和理论下界。仿真结果验证了提出的两种算法在不同的情况下都能有效地分配天线资源,得到有性能保证的系统收益。此项工作主要研究基站侧资源优化问题,可以进一步推广到毫米波通信其他星型结构网络应用中。
     \item 围绕基于单重网络外部性影响的移动群智感知系统平台成本最小化目标,提出了一种用于移动群智感知中隐私数据聚合的拍卖框架及激励机制。针对群智感知系统平台数据采集与移动用户隐私数据保护之间存在的矛盾关系提出了一种用户本地数据加噪的方案。其中用户所允许添加的噪声分布由感知平台设计决定,使得用户的数据隐私可以从差分隐私的量化指标上进行衡量。进一步,本文揭示了用户间所存在的负网络外部性效应并根据用户对于隐私保护级别的偏好,分成“消极隐私保护”和“积极隐私保护”两种场景进行问题的建模与求解。基于问题隐藏的单调性特性,针对两种场景分别提出了具有诚实性、个体理性、且计算高效的激励机制设计,使得感知平台可以近似地最小化感知平台的成本,同时满足对于聚合结果准确性的要求。本文通过理论分析结合充分的仿真实验验证了所提出的算法。

   \item 围绕基于双重网络外部性影响的隐私保护下频谱共享网络吞吐量最大化目标,提出了一种分布式信道选择接入机制。其中移动用户对信号传输功率水平添加随机扰动,以削弱基于RSS的潜在位置隐私攻击的威胁,而在制定频谱共享决策时同时考虑了其与其他用户在地理位置上和社交联系上的耦合关系。本文通过对于用户社交群体效用的定义刻画了社交效应(正网络外部性)和信号干扰(负网络外部性)影响下用户的实际效用,并将隐私保护下次级用户的信道共享问题建模成一个社交群体效用最大化(SGUM)博弈。进一步,本文针对性地提出了一个双时间尺度分布式学习算法。算法在较快时间尺度上进行效用学习,在较慢时间尺度上进行基于无悔学习的策略迭代。理论上可以证明提出的算法以概率1收敛至博弈的相关均衡集合。数值结果证实算法可以在隐私保护与网络吞吐量最大化的目标之间进行权衡。
%    \item 在以蜂窝基站为对象研究了星型毫米波通信网络之后,继续研究了毫米波通信在网状网络中的资源优化问题,场景设置为无线数据中心网络。随着数据流量特性的变化,数据中心网络中涌现的极度不平衡流量严重降低了网络传输效率。考虑在下一代数据中心网络中,利用毫米波高速无线传输代替传统服务器间的有线连接,以提高数据中心网络的灵活性及可扩展性等性能。利用60GHz毫米波天线阵列建立了服务器间点对点的直接通信,并优化调整不同任务流量所使用的通信资源以减少通信热点带来的通信性能下降。首先根据毫米波通信的特点,在每个服务器机架顶布置毫米波阵列天线,建立了单层和三层两种无线数据中心网络硬件布置方法与拓扑结构,并提出了网络中节点与边的生成方式。在此基础上提出了天线资源分配优化问题,通过变量替换等方式将其转化为几何规划问题并求解,所得算法降低了整个系统内所有任务流量的最大传输时延。仿真结果验证了所提出毫米波无线网络架构与算法的有效性。此项工作同时对系统内发射与接收双方进行资源优化管理,可以进一步推广到毫米波网状网络中的其他优化问题中。
   
  \item 围绕双重网络外部性的竞争性市场中数据服务提供商收益最大化目标,提出了基于两阶段斯塔克伯格博弈的数据定价机制。首先,揭示了该场景中的双重网络外部性效应,也就是移动用户的数据消费行为同时受到两方面因素的影响:社交效应( 正网络外部性)和拥塞效应(负网络外部性)。其次,将移动用户和服务提供商之间的策略互动,设计为一个两阶段的斯塔伯尔格博弈,分别由第一阶段的提供商博弈和第二阶段的用户博弈组成。针对用户博弈,本文刻画了均衡解的特征并建立了它的唯一性。对于提供商博弈,分析表明,对于提供商行为理性的情形以及提供商行为有限理性的情形,混合策略均衡解均是有保证的。进一步,本文提出了一种分布式学习算法,用于寻求提供商博弈的混合策略均衡解。最后,数值仿真结果对于正网络外部性和负网络外部性是如何影响系统性能的提供了见解,并验证了提供商有限理性行为对其收益所产生的负面影响。
  
 \item 探索技术与经济相结合的研究方法,创新性地从网络外部性效应的视角研究移动网络的效用优化问题。随着移动网络规模的快速增长,现代经济学所定义的网络外部性效应在移动网络中的表现日益凸显且典型,给网络效用和性能优化问题分析建模提出了更高要求。本文选择三个相对独立的移动网络场景,分别揭示了其中带有规律性的网络外部性影响,并针对性提出了网络效用优化方案。研究表明,深刻认识和准确把握移动网络的网络外部性特征,对于增强机制设计的可行性、提升效用优化的科学性是很必要的甚至是不可或缺的。同时,这也是对网络经济理论研究的延伸和丰富。此外,本文研究中运用的基础理论工具,既重视选用博弈论和机制设计及理性行为等前沿并成熟的理论模型,更注意把握了其在移动网络领域的实际适用性。
  
%  \item 研究了无线服务提供商收益最大化问题,其中移动用户的数据使用同时受到社交效应(正网络外部性)和拥塞效应(负网络外部性)的影响。为了刻画移动用户和服务提供商之间的交互,我们使用了斯塔克伯格博弈的问题建模并分析了其均衡解。特别地,我们首先求解了用户博弈的链接需求均衡,并证明了其唯一性特征。接着我们分别在完全理性服务提供商情形和有限理性服务提供商的情形下证明了服务提供商博弈中混合策略定价均衡的存在。我们的数值结果体现了正面网络效应和负面拥塞效应对系统性能的影响,以及有限理性行为对服务提供商收入造成的负面影响。
     
\end{enumerate}

随着移动网络规模的增长,网络外部性效应在大量移动网络场景中的影响日益凸显出来,为网络优化问题建模分析提出了更高的要求。本文聚焦于三个独立的移动网络场景,分析揭示了其中存在的网络外部性影响,并针对性地提出了有效的网络性能优化机制。同时,本文关注了移动网络中的潜在隐私泄露隐患,在网络性能优化机制设计中结合了隐私保护的措施,研究了隐私保护与性能优化之间的权衡。本文的研究成果为在真实移动网络场景中的性能优化设计提供了一定的指导意义。
%\rv{新的技术带来了丰富的计算与通信资源,也同时对资源的分配与应用提出了巨大的挑战。本文抓住了各种系统资源与通信时延及通信吞吐量等之间的关系,对基站中的计算资源、天线资源等进行优化,提升了系统性能。所得结果对毫米波无线通信系统中的资源分配与优化研究具有一定意义。

\section{研究展望}
本文由于可用时间和相关条件所限,现有的研究在前瞻性、实证性等方面还存有不尽人意的遗憾。同时,本文所研究的问题还存有进一步深入研究和探索的空间。同时,本文的一些成果对于其他一些移动网络性能优化问题场景也存在一定的借鉴意义价值。现将一些未来潜在的研究方向总结如下:

%本文针对下一代毫米波通信系统中的资源分配及优化展开研究,在毫米波无线网络的通信实时性、吞吐量提高等方面得到了一些新的结果。 毫米波通信本身涵盖了无线数字通信技术、信号处理技术、网络技术以及计算机技术等多个研究领域,仍有大量的问题亟待解决。就本文所研究的问题而言,可以在以下几个方面展开进一步研究:
\begin{enumerate}
%    \item 毫米波需要依靠波束成形技术应用在大规模天线阵列上,本文所研究的子阵列形状为均匀线性阵列。事实上有研究考虑矩形阵列、圆形阵列或其他异形阵列等平面阵列。如何将这些子阵列合理的布置在基站的天线阵列上以获取最优的系统收益将会是一个巨大的研究挑战。
%    \item 毫米波无线网络应用在数据中心网络系统中可以大大提高通信速率与网络的可扩展性,本文提出的单层网络或是三层网络拓扑结构事实上都处于同一个层级。在之后的研究中可以将Fat-tree等网络结构与本文所提网络结构结合,在架顶天线层级外再加入汇聚层或是核心层阵列天线进行转发,通过中继的方式在一定的路径跳数内提高服务器间无线通信范围。
%    \item 本文所研究的毫米波无线网络资源分配与优化主要是针对实时性、系统吞吐量以及公平性等目标,事实上还有其他优化目标例如可靠性、误码率、覆盖范围、频谱效率、能量效率以及多目标联合优化方案等可以进行优化设计。
%    \item 相较于软件仿真实验,硬件实现更能够反映算法在真实情况下的工作性能。因此,硬件试验平台构建是毫米波通信资源优化问题研究中的重要步骤。搭建毫米波通信实验平台并将优化算法应用在硬件平台上会是一项非常有意义的工作。
    \item 在移动群智感知平台成本最小化问题中,本文重点考虑了单一感知任务,且总用户集合固定不变的场景。当时间尺度拉大后,将出现多任务用户集合动态变化的情况。在这种更为复杂的情况下,如何对于任务进行分配,为用户提供长期的激励是一个有趣同时具有挑战的问题。此外,进一步的研究应当考虑构建一个保隐私群智感知测试平台,并基于平台上的系统实验对设计提出的激励机制进行验证和改进,以最大程度减小研究成果与实际应用效果之间的距离。
    \item 在频谱共享网络吞吐量最大化问题中,本文所使用的社交群体效用最大化框架(SGUM)对于其他一些移动网络中存在社交效应的问题场景具有一定的通用性,可以用于其他网络性能目标的优化问题中。此外,本文中的模型假设用户之间的社交联系强度为固定的且为用户所已知的信息。而在一些实际情况下,用户之间的社交联系强度可能带有一定的随机性,且需要用户通过社交学习(Social learning)的方式获得。这种情况下,如何设计有效的机制使系统达到对应的均衡状态是未来的一个潜在研究方向。
    \item 在移动网络服务提供商收益最大化问题中,本文使用了二次函数来刻画用户固有效用函数,使用了流量交叉乘积项来刻画用户之间的社交效应。进一步的研究可以去探索是否有更合适的函数形式可以被用来对用户的个体固有效用、用户间的网络外部性效应进行建模。另外,本研究中提供商的定价策略局限于静态定价模式且定价对于不同用户不进行区分,其好处是使得模型更为简洁,在实际中易于执行。而换用动态定价模式以及对于用户进行差异化定价是否可以在不付出太多机制复杂性代价的情况下提高系统的综合性能是值得深入探讨的问题。
    \item 本文所讨论的移动网络效用优化问题,涉及到网络中个体基于外界环境信息对于资源配置与定价策略的优化迭代。在实际中不断扩展的网络规模与与频繁变化的环境信息始终会是该类问题的主要挑战。而当前机器学习领域以深度强化学习为代表的一系列前沿研究成果对于分布式、自适应调度机制的研究设计有着巨大的潜在应用价值。因此,机器学习技术与移动网络效用优化的一些深度结合将会是未来的另一个重要研究方向。
\end{enumerate}







