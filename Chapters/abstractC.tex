\begin{abstract}

%    随着无线通信技术的不断发展,无线应用逐渐进入生产生活中的各个领域。人们对无线通信系统的性能需求也逐渐提升,这主要包括更高的通信带宽、更低的通信时延、更稳定的通信连接、更大规模的通信容量以及更高的通信效率。然而,现有的无线通信系统难以达到以上要求,人们迫切盼望着下一代无线通信系统的到来。
    21世纪的人类进入了以网络为核心的信息时代。移动网络以其泛在、便捷、智能的突出特点和优势,已成为人们学习、工作、生活离不开的新空间和经济社会发展最具生机活力的新领域。与此同时,随着移动网络的工具属性、社交属性、媒体属性、平台属性越来越凸显,人们对移动服务质量的要求日益增高、对隐私保护和安全风险的担忧日益加深。因此,在依靠5G的全面商用以提升移动网络技术性能的同时,针对移动网络运行中客观存在的资源有限性、网络参与者的个体自利性、提升网络性能与数据隐私保护的矛盾性和网络外部性效应的典型性等特征,深入研究创新移动网络的机制设计、优化移动网络的整体效用显得愈发重要。
    
    研究者们针对移动网络中的性能效用优化及机制设计研究仍存有多方面不足。比如:在解决网络中效用优化的问题时缺乏从网络外部性角度的深入分析和讨论;对于优化网络性能与用户隐私保护之间权衡的考量与刻画不够充分;现有激励机制设计中对于独立个体在交互中的策略性行为假设过于理想使实际效果存在明显局限等。
     
     本文在国内外相关问题现有研究的基础上,围绕提升移动网络运行中的主体效用,以网络外部性为重要视角,运用博弈论与机制设计理论相关工具,对移动网络中群智感知、频谱共享、数据服务三个基本场景中的网络效用优化问题进行了基础理论层面的研究,相应提出了保隐私激励机制、保隐私频谱共享机制、数据定价机制的方案及协议算法。本文的研究成果丰富了移动网络效用优化理论,对移动网络机制设计实践具有指导意义。本文的主要工作和贡献及创新之处有:

%    随着移动通信技术的快速发展、智能移动终端全球范围内的大规模普及以及移动互联网应用产业的兴起,近些年来移动用户和移动数据流量呈现出爆炸性的指数增长。人们日益增长的对于数据流量的需求和对于服务质量的要求不断地给移动网络技术的发展带来新的挑战。除了依靠日趋成熟的5G技术在网络基础架构能力的提升以外,网络服务运营商还可以通过对有限频谱资源的高效利用以及对数据服务定价机制的合理设计等方式有针对性地对供给侧的资源进行合理调度,实现网络整体性能的优化。另一方面,随着移动设备感知能力的愈发强大,移动网络逐渐承载起以移动用户为核心的感知网络,成为用户个人数据的主要生成来源、用户周边环境数据的主要采集渠道。其一方面推动了以移动群智感知技术为代表的移动大数据技术发展,同时却也引发了人们对于移动网络中个人隐私保护的密切关注。
%
%    网络外部性效应是网络经济学领域的主要研究对象之一,广义上指的是一个网络中任意个体所获得的效用受网络中其他个体的影响。近年来移动网络规模上的不断扩张使得网络外部性效应在一些移动网络的问题场景中凸显出来。
%    %这些影响可能直接取决于网络的规模大小,抑或间接取决于其他网络个体的参与程度以及个体之间关联的强弱。
%%    在这里,正向网络外部性指个体用户的效用随着网络中其余用户参与程度的增强而提高,而相应的负向网络外部性则对应于个体移动用户效用的降低。在有些场景中甚至会出现多重网络外部性的复合效应。
%    具体来说,单一用户从移动网络中获取的效用可以随着网络中其余用户参与程度的增强而提高(正网络外部性)或降低(负网络外部性)。由于网络外部性效应对于网络中个体效用的显著影响,在移动网络机制设计与性能优化中对网络外部性效应进行充分考量是至关重要的。
%    %另一方面,由于移动设备与用户的紧密关联,移动网络应用中的隐私保护问题逐渐成为了人们讨论的焦点,而在一些场景中移动网络设计者不得不面对隐私保护与网络性能优化目标上的冲突。
%    近些年,研究者们在以移动网络性能优化和隐私保护为目标的机制设计方面取得了众多研究进展,然而现有工作仍然存在以下三方面的不足:1)在解决网络性能优化的问题时缺乏从网络外部性角度的分析与讨论,缺乏对于网络外部性影响的深刻理解;2)网络性能优化与用户隐私保护相结合的研究分析还较少,尤其缺乏针对性能-隐私权衡的刻画与考量;3)现有激励机制设计中对于用户个体在交互中的策略性行为假设较为理想,缺乏对于个体行为理性模型差异性的考量。本文结合国内外研究现状,对频谱共享、移动群智感知、移动数据服务三个基本场景中的网络性能优化问题进行了探索和改进,具体包括以下几部分工作:
    \begin{enumerate}
        \item 围绕基于单重网络外部性影响的隐私保护下移动群智感知系统平台成本最小化目标,针对群智感知系统平台数据采集与移动用户隐私数据保护之间事实上存在的利益关系矛盾,提出了用于移动群智感知中隐私数据聚合的拍卖框架。首先,鉴于移动用户的感知能力和提供数据的隐私成本存在的差异性,本文设计了用于隐私保护的数据噪声分布和本地差分隐私加噪方案,使得数据聚合结果达到准确性指标。其次,本文揭示了该数据加噪方案下用户间所存在的{\kaishu 负网络外部性}效应,并根据用户对于隐私保护级别的偏好分成“消极隐私保护”和“积极隐私保护”两种情形进行问题的建模与求解。第三,基于隐藏的平台成本优化问题单调性特性,本文分别针对两种情形设计了具有诚实性、个体合理性,计算高效性的激励机制(DPDA和EDPDA),使感知平台可以在满足对于聚合结果准确性要求的条件下以近似最优的成本购得用户的数据。最后,本文通过理论分析结合充分的仿真实验验证了所提方案的可行性,表明可以有效权衡群智感知系统平台采集数据与移动用户隐私保护的“两难”问题。

%	\item 研究了隐私保护下群智感知系统平台成本最小化问题,提出了一种用于移动群智感知中隐私数据聚合的拍卖框架。由于移动用户感知能力和提供数据的隐私成本的差异性,作为拍卖者感知平台的挑战在于选择合适的参与用户,针对性地为他们设计数据噪声分布,使得用户的数据隐私得到保护,同时数据聚合结果达到既定准确性指标。本文提出了一种用户本地数据加噪的方案,并揭示了用户间对于隐私保护级别的{\kaishu 负网络外部性}效应。根据用户对于隐私保护级别的偏好,本文分成“消极隐私保护”和“积极隐私保护”两种场景进行问题的建模与求解。基于问题隐藏的单调性特性,针对两种场景分别提出了具有诚实性、个体合理性、计算高效性的激励机制设计DPDA和EDPDA,使得感知平台在满足结果准确性约束的同时可以给予用户足够的奖励并近似地实现成本最小化。本文通过理论分析结合充分的仿真实验验证了所提出的算法的表现。
     
        \item 围绕基于双重网络外部性影响的隐私保护下频谱共享网络吞吐量最大化目标,提出了一种分布式信道选择接入机制。首先,针对基于接收信号强度(RSS)的位置隐私攻击,本文提出了一种信号传输功率扰动的方案以降低隐私攻击的有效性。其次,揭示了移动用户在频谱共享决策制定中同时考虑了其在地理位置上和社交联系上与其他用户的耦合关系。这两层耦合关系分别决定了其所受周围用户的信号干扰影响({\kaishu 负网络外部性})以及其所受其他用户的社交效应({\kaishu 正网络外部性})。第三,本文使用社交群体效用刻画了用户在两种网络外部性影响作用下所获得的实际效用,将隐私保护下次级用户的频谱共享问题建模成一个社交群体效用最大化(SGUM)博弈。第四,针对采用的博弈模型本文设计了一个基于无悔学习规则的双时间尺度分布式学习算法,并证明其可以收敛至博弈的相关均衡集合。数值仿真结果证实所提出的算法的有效性。
%        \item 研究了毫米波通信应用在蜂窝小区网络中的基站资源优化分配问题。考虑基站引入大规模多输入多输出天线阵列,当已知小区内多个用户位置的情况下,如何分配其天线资源以最大化系统收益即系统吞吐量的问题。将基站上的大规模天线阵列虚拟地分成若干个均匀线性子阵列,分别通过波束成形技术与对应的用户进行通信。由于大规模天线阵列与每个均匀线性子阵列都有着固定的形状,因此本文将资源优化问题建模为如何在矩形阵列中设计并合理地放置这些子阵列使得系统的收益最大。这既需要考虑每个用户对应子阵列的天线数量,又需要考虑子阵列在矩形阵列中的位置。根据子阵列的放置方式不同又包含两种的情况,1)所有线性子阵列都平行于矩形阵列的一条边;2)为了进一步提高系统收益,线性子阵列可以相互垂直放置。本文对以上两种情况分别建模,解决了两个NP-hard问题,得到了两种情况下多项式时间的近似算法。仿真结果验证了提出的两种算法在不同情况下都能有效地分配天线资源,得到有性能保证的系统收益。            
        
 %       \item 研究了毫米波通信应用在无线数据中心网络中的资源优化分配问题。考虑在数据中心网络中利用毫米波无线连接代替传统服务器间的有线连接,以提高数据中心网络灵活性及可扩展性的问题。随着数据应用不断的发展,数据中心网络中涌现了极度不平衡流量。为了降低不平衡流量带来的网络拥堵,在每个服务器机架顶布置毫米波阵列天线以建立点对点无线数据中心网络。通过分析阵列天线与波束成形技术特点,本文提出了无线数据中心网络硬件实现方式以及单层和三层两种网络拓扑结构,并给出了网络拓扑中节点与边的生成方式。之后,为了降低给定时间内系统中任务流的最大完成时间,本文建模了天线资源分配优化问题,通过变量替换等方式将其转化为几何规划问题并给出了解法。仿真结果验证了提出的毫米波无线数据中心网络结构与资源优化算法的有效性。
       \item 围绕双重网络外部性的竞争性市场中数据服务提供商收益最大化目标,提出了基于两阶段斯塔克伯格博弈的数据定价机制。首先,揭示了该场景中存在的双重网络外部性效应,即移动用户的数据消费行为同时受到两方面因素的影响:社交效应({\kaishu 正网络外部性})和拥塞效应({\kaishu 负网络外部性})。其次,将移动用户和服务提供商之间的策略互动,设计为一个由第一阶段的提供商博弈和第二阶段的用户博弈组成的两阶段的斯塔伯尔格博弈模型。针对用户博弈,本文刻画了均衡解的特征并建立了它的唯一性。对于提供商博弈,分析表明,对于提供商行为理性的情形以及提供商行为有限理性的情形,混合策略均衡解均是有保证的。进一步,本文提出了一种分布式学习算法,用于寻求提供商博弈的混合策略均衡解。最后,数值仿真结果对于正网络外部性和负网络外部性是如何影响系统性能的提供了见解,并验证了提供商有限理性行为对其收益所产生的影响。
       
       
       %研究了服务提供商收益最大化问题。在竞争性数据服务市场中多个服务提供商通过定价策略的制定以最大化自身收益。市场中移动用户的数据消费行为同时受到社交效应({\kaishu 正网络外部性})和拥塞效应({\kaishu 负网络外部性})两方面因素的影响。为了刻画移动用户和服务提供商之间的策略互动,本文使用斯塔克伯格博弈模型对问题建模,具体包括第一阶段的提供商博弈和第二阶段的用户博弈。针对用户博弈,本文刻画了均衡解的特征并证明了其唯一存在性。针对提供商博弈,本文考虑提供商行为理性与行为有限理性两种场景,并证明了博弈中混合策略均衡解的存在性。进一步,本文提出了一种分布式学习算法,用于寻找提供商博弈的混合策略均衡解。数值仿真结果体现了网络效应和拥塞效应对于系统性能的影响,并验证了有限理性行为对于服务提供商收益的负面影响。
       
       \item 探索技术与经济相结合的研究方法,创新性地从网络外部性效应的视角研究移动网络的效用优化问题。本文选择三个相对独立的移动网络场景,分别揭示了其中带有规律性的网络外部性影响,并针对性提出了网络效用优化方案。结果表明,深刻认识和准确把握移动网络的网络外部性特征,对于增强机制设计的可行性、提升效用优化的科学性是很必要的甚至是不可或缺的。同时,本文的工作也是对网络经济理论研究的延伸和丰富,本文所运用的博弈论与机制设计的理论工具,一方面对相关领域的前沿技术与成熟理论模型进行了结合,另一反面也注重了其在移动网络领域的实际适用性。 

       % \item 对全文进行了总结,对未来的有关研究工作进行了展望,提出了需要进一步研究的问题。
    \end{enumerate} 
    
\keywords{移动网络;效用优化;机制设计;网络外部性;博弈论}
\end{abstract}
