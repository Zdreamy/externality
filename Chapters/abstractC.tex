\begin{abstract}

%    随着无线通信技术的不断发展,无线应用逐渐进入生产生活中的各个领域。人们对无线通信系统的性能需求也逐渐提升,这主要包括更高的通信带宽、更低的通信时延、更稳定的通信连接、更大规模的通信容量以及更高的通信效率。然而,现有的无线通信系统难以达到以上要求,人们迫切盼望着下一代无线通信系统的到来。
    随着移动通信设备的普及,以及移动社交网络、媒体娱乐等相关领域的飞速发展,全球移动用户及数据流量近些年来一直保持着大幅增长的趋势。人们对于通信、数据服务质量上日益增高的要求不断给无线网络技术的发展带来新的挑战。除了依靠下一代5G技术在通信网络基础架构能力的提升外,网络数据运营商还可以通过对有限频谱资源的高效利用、对数据服务定价机制的合理设计等方式有针对性地基于需求协调供给侧的有限资源,实现网络整体性能的优化。另一方面,随着移动设备感知能力的愈发强大,移动网络俨然已经成为第三方应用对用户个人数据、城市环境数据采集的主要渠道。其一方面推动了包括移动群智感知技术在内的移动大数据技术领域的发展,而同时却也引发了人们对于个人隐私泄露的担忧。

    网络外部性效应是网络经济学领域的主要研究对象之一。简单来说,其指的是一个网络中任意个体所获得的效用所受网络中其他个体的影响。这种影响可能简单取决于网络的规模大小,有时也取决于其他个体的参与程度以及个体之间的关联的强弱。近年来移动网络规模上的不断扩张使得网络外部性效应在一些移动网络的问题场景中凸显出来。
%    在这里,正向网络外部性指个体用户的效用随着网络中其余用户参与程度的增强而提高,而相应的负向网络外部性则对应于个体移动用户效用的降低。在有些场景中甚至会出现多重网络外部性的复合效应。
    具体来说,单一用户从移动网络中获取的效用可以随着网络中其余用户参与程度的增强而提高(正网络外部性)或降低(负网络外部性)。由于网络外部性效应对于网络中个体效用的显著影响,在移动网络机制设计与性能优化中对网络外部性效应进行充分考量是至关重要的。
    %另一方面,由于移动设备与用户的紧密关联,移动网络应用中的隐私保护问题逐渐成为了人们讨论的焦点,而在一些场景中移动网络设计者不得不面对隐私保护与网络性能优化目标上的冲突。
    近些年,研究者们在以移动网络性能优化和隐私保护为目标的机制设计方面取得了众多研究进展,然而现有工作仍然存在以下三方面的不足:1)在解决网络性能优化的问题时缺乏从网络外部性角度的分析与讨论,缺乏对于网络外部性影响的深刻理解;2)网络性能优化与用户隐私保护相结合的研究分析还较少,尤其缺乏针对性能-隐私权衡的刻画与考量;3)现有激励机制设计中对于用户个体在交互中的策略性行为假设较为理想,缺乏对于个体行为理性模型差异性的考量。本文结合国内外研究现状,对频谱共享、移动群智感知、移动数据服务三个基本场景中的网络性能优化问题进行了探索和改进,具体包括以下几部分工作:
    \begin{enumerate}
        \item 概述了无线通信领域网络外部性的研究背景,并综述了本文涉及到的移动网络机制设计及隐私保护的相关研究进展。

%        \item 研究了在毫米波无线通信系统中多用户快速定位与跟踪问题。毫米波通信需要发射与接收双方通过波束对准,利用极窄的波束建立连接。例如在蜂窝小区场景中,基站需要对多个用户位置信息进行实时准确的估计与跟踪,以建立低时延的毫米波方向性通信。现有的多目标跟踪算法,如概率假设密度粒子滤波算法,难以做到流水线化计算,制约了跟踪实时性的进一步提高。对此,本文提出了一种改进的重采样算法,通过引入粒子复制序列集合以及待复制粒子序列,使得需要重采样的粒子在只获得自身信息与之前重采样粒子信息时即可进行粒子复制运算。通过引入该改进的重采样算法,整个概率假设密度粒子滤波算法能够实现完全流水线化运算。在此基础上,为进一步降低运行时延,在多核处理器硬件平台上提出了计算资源优化分配问题,并通过解决一组混合整数规划问题,得到了高时效的近似解法。仿真结果验证了提出算法在保证跟踪精度的情况下,能显著降低整个滤波算法的计算时延,有效提高多用户跟踪的实时性,为毫米波通信的建立奠定了基础,并为基站与每个用户的低时延通信提供了保证。

	\item 研究了隐私保护下群智感知系统平台成本最小化问题,提出了一种用于移动群智感知中隐私数据聚合的拍卖框架。由于移动用户感知能力和提供数据的隐私成本的差异性,作为拍卖者感知平台的挑战在于选择合适的参与用户,针对性地为他们设计数据噪声分布,使得用户的数据隐私得到保护,同时数据聚合结果达到既定准确性指标。本文提出了一种用户本地数据加噪的方案,并揭示了用户间对于隐私保护级别的{\kaishu 负网络外部性}效应。根据用户对于隐私保护级别的偏好,本文分成“消极隐私保护”和“积极隐私保护”两种场景进行问题的建模与求解。基于问题隐藏的单调性特性,针对两种场景分别提出了具有诚实性、个体合理性、计算高效性的激励机制设计DPDA和EDPDA,使得感知平台在满足结果准确性约束的同时可以给予用户足够的奖励并近似地实现成本最小化。本文通过理论分析结合充分的仿真实验验证了所提出的算法的表现。
     
        \item 研究了隐私保护下具有社交意识的网络吞吐量最大化问题。在数据库辅助频谱共享的场景中研究了移动用户位置隐私保护下的频谱接入机制。其中移动用户在频谱共享决策制定中同时考虑了其在地理位置上和社交联系上与其他用户的耦合关系。这两层耦合关系分别决定了其所受周围用户的信号干扰影响({\kaishu 负网络外部性})以及其所受其他用户的社交效应({\kaishu 正网络外部性})。针对基于接收信号强度(RSS)的位置隐私攻击,本文提出了一种信号传输功率扰动的方案以降低隐私攻击的有效性。进一步,本文将隐私保护下次级用户的频谱共享问题建模成一个社交群体效用最大化(SGUM)博弈。其中次级用户以最大化其社交群体效用为目标进行策略性的信道接入选择。而社交群体效用即刻画了用户在两种网络外部性影响作用下所获得的实际效用。针对该博弈模型本文设计了一个基于无悔学习规则的双时间尺度分布式学习算法,并证明其以概率1收敛至博弈的相关均衡集合。数值仿真结果证实所提出的算法可以在隐私保护与网络吞吐量最大化的目标之间进行权衡。
%        \item 研究了毫米波通信应用在蜂窝小区网络中的基站资源优化分配问题。考虑基站引入大规模多输入多输出天线阵列,当已知小区内多个用户位置的情况下,如何分配其天线资源以最大化系统收益即系统吞吐量的问题。将基站上的大规模天线阵列虚拟地分成若干个均匀线性子阵列,分别通过波束成形技术与对应的用户进行通信。由于大规模天线阵列与每个均匀线性子阵列都有着固定的形状,因此本文将资源优化问题建模为如何在矩形阵列中设计并合理地放置这些子阵列使得系统的收益最大。这既需要考虑每个用户对应子阵列的天线数量,又需要考虑子阵列在矩形阵列中的位置。根据子阵列的放置方式不同又包含两种的情况,1)所有线性子阵列都平行于矩形阵列的一条边;2)为了进一步提高系统收益,线性子阵列可以相互垂直放置。本文对以上两种情况分别建模,解决了两个NP-hard问题,得到了两种情况下多项式时间的近似算法。仿真结果验证了提出的两种算法在不同情况下都能有效地分配天线资源,得到有性能保证的系统收益。            
        
 %       \item 研究了毫米波通信应用在无线数据中心网络中的资源优化分配问题。考虑在数据中心网络中利用毫米波无线连接代替传统服务器间的有线连接,以提高数据中心网络灵活性及可扩展性的问题。随着数据应用不断的发展,数据中心网络中涌现了极度不平衡流量。为了降低不平衡流量带来的网络拥堵,在每个服务器机架顶布置毫米波阵列天线以建立点对点无线数据中心网络。通过分析阵列天线与波束成形技术特点,本文提出了无线数据中心网络硬件实现方式以及单层和三层两种网络拓扑结构,并给出了网络拓扑中节点与边的生成方式。之后,为了降低给定时间内系统中任务流的最大完成时间,本文建模了天线资源分配优化问题,通过变量替换等方式将其转化为几何规划问题并给出了解法。仿真结果验证了提出的毫米波无线数据中心网络结构与资源优化算法的有效性。
       \item 研究了服务提供商收益最大化问题。在竞争性数据服务市场中多个服务提供商通过定价策略的制定以最大化自身收益。市场中移动用户的数据消费行为同时受到社交效应({\kaishu 正网络外部性})和拥塞效应({\kaishu 负网络外部性})两方面因素的影响。为了刻画移动用户和服务提供商之间的策略互动,本文使用斯塔克伯格博弈模型对问题建模,具体包括第一阶段的提供商博弈和第二阶段的用户博弈。针对用户博弈,本文刻画了均衡解的特征并证明了其唯一存在性。针对提供商博弈,本文考虑提供商行为理性与行为有限理性两种场景,并证明了博弈中混合策略均衡解的存在性。进一步,本文提出了一种分布式学习算法,用于寻找提供商博弈的混合策略均衡解。数值仿真结果体现了网络效应和拥塞效应对于系统性能的影响,并验证了有限理性行为对于服务提供商收益的负面影响。 

        \item 对全文进行了总结,并对进一步的研究工作进行了展望。
    \end{enumerate} 
    
\keywords{网络外部性;网络性能优化;机制设计;隐私保护;博弈论}
\end{abstract}
