\begin{thanks}
从高中保送到浙大预习班到现在已十余年,我在求是园中留下了无数美好回忆。尤其幸运的是,本科毕业后我遇到了一位年轻有为、远见卓识的导
师,投身于一个朝气蓬勃、催人奋进的科研团队。感谢所有关心、支持和帮助我的师长、同窗与亲友们,我会一一铭记并永存感激。

谨向孙优贤院士致以崇高的敬意和衷心的感谢! 您渊博的学识、求真务实的态度、对事业的不断追求和对我深切的关爱,都深深地影响和教育了我,并将继续激励我不断前行。您殷切的期望与鼓励为我今后的工作点亮了一盏指路明灯!

衷心感谢导师陈积明教授细致入微的关怀和指导! 陈教授为我开启了学术科研的大门,在培养我缜密的逻辑思维与严谨的科研态度的
同时,也让我不断积极向上的心态。感谢您不仅指导我如何进行科研工作,还教会了我生活中如何做人、如何做事,更感谢您在我低谷期所给予的宽容与支持。您开放活跃的思维、锐意进取的态度、求真务实的精神为我们年轻一辈树立了榜样,并将不断指引我们继续前进!

非常感激林庆老师在工作、学习和生活等各个方面给予我的热情关爱和细心帮助!您是我在杭州最亲密最敬爱的长辈之一,您对我和我家人无微不至的支持与鼓励,是阴雨绵绵的杭城最温暖的阳光。

感谢California Institute of Technology的Steven H. Low教
授,Singapore University of Technology and Design的David Yau教授,浙江大学的史治国教授、程鹏教授、贺诗波研究员、吴均峰研究员,你们扎实的理论功底对我科研工作提出的宝贵意
见,和你们的每次讨论都让我受益匪浅。感谢杨秦敏教授、李凤旺老师、朱耕宇老师、黄懿明老师等各位
老师的关心与帮助。感谢傅凌焜师兄对我的特别帮助,感谢张永敏、史秀纺、周成伟、赵成成、李松原、杨泽域、张梦源、刘昊俣、李可汉等在我学术科研与生活上给予
的无私帮助,感谢浙江大学控制科学与工程学院网络传感与控制研究组(NeSC-IIPC)的每一位在读或业已毕业的成员,非常开心能与你们一起度过这段人生最美好的时光。

最后,要特别感谢我的父母和亲人,是你们对我无私的爱与支持,才让我可以在举目无亲的杭州一个人坚持走到现在。

%感谢张帆女士,从儿时的玩伴到伴侣,感谢你在我博士生涯最后阶段对我的宽容与鼓励,希望我们能一路相伴。

回想多年的求学经历,需要感谢的人太多太多。正是有家人、师长、同学和朋友们的关心帮助,才能顺利完成学业。展望未来的人生道路,要走的路还很长。我一定继续努力,以更大的成绩回报祖国,回报浙江大学,回报大家的信任。

谨以此文献给所有关心和帮助过我的亲朋良师。

	\begin{flushright}
		{李超}
		
		{二〇一九年三月于求是园}
	\end{flushright}
\end{thanks}
