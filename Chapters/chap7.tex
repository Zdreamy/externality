\chapter{beamer的使用}


做好了论文,也要准备答辨的PPT了。当然用PowerPoint做是没有任何问题的。
演示也可以用\LaTeX{}做的,这一章就介绍用beamer模版做一个演示用的PDF版PPT。

\section{模版工具包}

首先,新建一个\TeX{}文件,使用beamer这个cls包,为使用中文,还需要CTeX扩展包,如果在演示中使用表格,还需要有booktabs包,multirow包等等。
beamer有一些整体的风格模版(theme),有Warsaw,Madrid,Pittsburgh,Rochester,等以城市命名的若干模版,这些模板的式样可以在/usr/share/texmf/doc/latex/beamer目录中找到相应的例子,根据例子选择合适的模版即可。这里使用的是Warsaw模版。
相关的代码如下所示:

{
\linespread{1}
\zihao{5}\noindent
\begin{verbatim}
\documentclass{beamer}
\usepackage{ctex}
\usepackage{booktabs}
\usepackage{multirow}
\usepackage{upgreek}
\usetheme{Warsaw}
\end{verbatim}
}

\section{标题及作者等}

与论文略有不同的是,在\verb+\begin{document}+之前,就需要对这个演示的标题及作者以及日期进行相应的设置。相关代码如下:

{
\linespread{1}
\zihao{5}\noindent
\begin{verbatim}
\title{题目第一行\\题目第二行}
\author[WANG]{王东举}
\date{2015年6月12日}
\end{verbatim}
}

关于\verb|\author[WANG]{王东举}|中的WANG是一个出现在演示每页下面名字的简写。这个参数也可以不写。

\section{正文部分}

在例行的\verb|\begin{document}|命令之后,就进入到了演示页的正文部分。
beamer文档的正文的主要组成部分是frame环境,
每个frame就是一页,在frame内部,写法与论文正文的写法相似,可以有文字、列举(enumerate)环境,图形,表格,小页等。
在frame之外,可以有section,subsection这样的分章节命令,相应命令给出的章节表,会自动同步到目录以及每页页头上去。

\subsection{题目页}

首先生成题目题目页,相应的代码如下:

{
	\linespread{1}
	\zihao{5}\noindent
	\begin{verbatim}
	\begin{document}
	\begin{frame}
	\titlepage
	\end{frame}
	\end{verbatim}
}

\subsection{目录页}

目录可以自动生成,相应的代码如下:

{
	\linespread{1}
	\zihao{5}\noindent
	\begin{verbatim}
	\frametitle{目录}
	\tableofcontents
	\end{frame}
	\end{verbatim}
}

\subsection{正文页}

在写正文frame之前,相应的section,subsection题目要写在frame环境之外。相应代码如下:

{
	\linespread{1}
	\zihao{5}\noindent
	\begin{verbatim}
	\section{第一章}
	\begin{frame}
	\frametitle{绪论}
	\framesubtitle{小标题}
	这是frame里面的正文部分
	\begin{enumerate}
	\item{}内容1
	\item{}内容2
	\item{}内容3
	\end{enumerate}
	\end{frame}
	\end{verbatim}
}

\subsection{block环境}

beamer中的block环境可以起到醒目作用,block的颜色风格与整个演示的风格是相符的。下面是一个例子。

{
	\linespread{1}
	\zihao{5}\noindent
	\begin{verbatim}
	\section{第二章}
	\begin{frame}
	\frametitle{block环境}
	\framesubtitle{小标题}
	这是frame里面的正文部分
	\begin{block}{这是一个block}
	这个block可以用来写一些定义等。
	\end{block}
	\end{frame}
	\end{verbatim}
}


\subsection{插入图片}

在frame中插入图片与在正文中插入图片的方式相同。如下面代码为例:

{
	\linespread{1}
	\zihao{5}\noindent
	\begin{verbatim}
\begin{frame}
	\frametitle{一个图片的例子}	
	\includegraphics[width = 0.6\textwidth]{../Pictures/LHS.eps}
\end{frame}
	\end{verbatim}
}

也可以既有文字又不图片,做一个相应的排布,这个地方利用columns这个环境对frame进行分列。相应代码如下:

{
	\linespread{1}
	\zihao{5}\noindent
	\begin{verbatim}
\begin{frame}
	\frametitle{columns分列}	
	\begin{columns}
		\begin{column}{0.5\textwidth}			
		\includegraphics[width = 0.8\textwidth]{../Pictures/LHS.eps}
		\end{column}
		\begin{column}{0.5\textwidth}			
			\begin{enumerate}
			\item{}内容1
			\item{}内容2
			\item{}内容3
			\end{enumerate}			
		\end{column}
	\end{columns}	
\end{frame}
	\end{verbatim}
}

注意上面代码中相应比例宽度的设置。每列0.5倍的文字宽,在每个column中,相应的textwidth是这个column的宽度。

\subsection{插入表格}

在frame中插入表格与在正文中插入表格的方式相同。如下面代码为例:

{
	\linespread{1}
	\zihao{5}\noindent
	\begin{verbatim}
	\begin{frame}
	\frametitle{一个表格的例子}	
	\begin{table}[htb]
	\centering
	\begin{tabular}[t]{c|l|r|p{4cm}}
	\hline
	居中 & 靠左 & 靠右 & 靠左宽4cm\\
	\hline
	Center & Left & Right & Width=4cm\\
	\hline
	\end{tabular}
	\end{table}
	\end{frame}
	\end{verbatim}
}

\section{动态效果}

beamer还可以实现在同一页中的不同几项依次出现的效果。
实现方式是利用pause命令,相应代码如下:

这是一个举例,下面利用itemize来进行列举。

{
	\linespread{1}
	\zihao{5}\noindent
	\begin{verbatim}
	\begin{frame}
	\frametitle{一个依次出现的例子}		
	\begin{itemize}
	\item 这是第一个例子
	\pause
	\item 这是第二个例子
	\pause
	\item 这是第三个例子
	\end{itemize}
	\end{frame}
	\end{verbatim}
}
